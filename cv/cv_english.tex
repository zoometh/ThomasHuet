\documentclass[11pt]{report} %%% use \documentstyle for old LaTeX compilers


\usepackage{geometry}
\geometry{legalpaper, margin=1in}
%\usepackage[utf8]{inputenc}
%\usepackage[english]{babel}
\usepackage{hyperref}
\usepackage{graphicx}
\graphicspath{{../img/}}

\setlength\parindent{0pt} % noindet


\begin{document}

Thomas HUET (PhD), Associate Researcher \\
\href{https://archimede.cnrs.fr/index.php/annuaire/123-annuaire/e-h/456-thomas-huet}{LabEx ARCHIMEDE, UMR 5140 ASM-CNRS, Universit\'{e} Paul-Val\'{e}ry Montpellier 3, France} \\
tel: (+0034) 693 726 091 \\
mail: \href{mailto:thomashuet7@gmail.com}{thomashuet7@gmail.com} \\
ORCID: \href{https://orcid.org/0000-0002-1112-6122}{0000-0002-1112-6122} \\
GitHub: \href{https://github.com/zoometh/thomashuet.github.io/blob/main/README.md}{zoometh} \\
Google Scholar: \href{https://scholar.google.fr/citations?user=2hKEVaIAAAAJ}{2hKEVaIAAAAJ} \\
ResearchGate: \href{https://www.researchgate.net/profile/Thomas\_Huet2}{Thomas\_Huet2} \\
 
\section*{SKILLS}

Computational Archaeology (programming languages, GIS databases), scientific writing, fluent in French, English and Spanish. 

\section*{ACADEMIC AND PROFESSIONAL POSITIONS}

\textbf{2020 }Responsible of the survey of the medieval engravings of Sauri (Catalonia, Spain), 2${}^{nd\ }$campaign, Universitat Aut\'{o}noma de Barcelona (UAB), Spain, 19 October ---24 October.\textbf{}
\smallbreak
\textbf{2019-20 }Researcher at Archa\"{i}os company, in charge of the database integration for the survey of Al-'Ula, Saudi-Arabia (Archa\"{i}os, AFALULA, RCU). This included field engineering management of the oasis survey data (Al-'Ula, Saudi-Arabia).
\smallbreak
\textbf{2019 }Responsible of the survey of the medieval engravings of Sauri (Catalonia, Spain), 1${}^{st\ }$campaign, Universitat Aut\'{o}noma de Barcelona (UAB), Spain, 15 September---15 October.\textbf{}
\smallbreak
\textbf{---  }Project designer for the creation of a technological hub in Emerging technologies for Humanities, UMR 8546 CNRS/PSL --- AOrOc, Paris, resp. K. Gruel. Project granted.
\smallbreak
\textbf{2018 }Research Associate (IR), UMR 7264 CEPAM---CNRS, Universit\'{e} Nice Sophia-Antipolis, project \textit{C\'{e}ramiques Imprim\'{e}es de M\'{e}diterran\'{e}e occidentale} (CIMO), resp. D. Binder, 1 May-30 August and 1 October---30 November.
\smallbreak
\textbf{---  }Research Associate (IR), UMR 5140, ASM-CNRS, Universit\'{e} Paul Val\'{e}ry Montpellier 3, project \textit{EpiSpat} (aka \textit{ArchaEpigraph}), LabEx ARCHIMEDE, resp. C. Pellecuer, 1 April-1 May and 1-30 September.
\smallbreak
\textbf{2017 } Development of the Information system (GIS, datatabase, topography and photogrammetry), grotte de Pertus II, Alpes-de-Haute-Provence, UMR 7264 CEPAM-CNRS, Universit\'{e} Nice Sophia-Antipolis and Ev\'{e}ha, resp. C. Lep\'{e}re, 7-27 July.
\smallbreak
\textbf{2016 } Development of the information system (GIS, database, topography and photogrammetry) of Pertus II cave, Alpes-de-Haute-Provence, dir. C. Lep\'{e}re, 1-15 August 2016.
\smallbreak
\textbf{2015-16  }Postdoctoral fellow, LabEx ARCHIMEDE, UMR 5140 ASM-CNRS and University Paul-Val\'{e}ry, Montpellier. Project: ''Study of Final Bronze Age figurative ceramic decorations in South France and Northeastern Spain'', 1 October 2015-- 30 September 2016.
\smallbreak
\textbf{2015 }Auditor in the research seminar "Regards crois\'{e}s sur la notion de paysage", lector~: S.~Robert, EHESS, March --- June.\textbf{}
\smallbreak
\textbf{2014  }Research\textbf{ }Associate (IR), project \textit{Archaepigraph} project, UMR 6249 Chrono---Environnement, Universit\'{e} de Franche---Comt\'{e}, dir. M.-J. Ouriachi and L. Nuninger, 1 September --- 30 October.
\smallbreak
\textbf{2013-14  }Research Assistant (IE), USR CNRS -- UB 3516, GeoBFC platform, MSH de Dijon, Universit\'{e} de Bourgogne, project OH-FET (Historical Object, Function, Space, Time), dir. L. Saligny, 1 June-31 May.

\section*{EDUCATION}

\textbf{2013 }PhD prize for publication. CASDEN -- Banque Populaire (UFR LSH, Universit\'{e} Nice Sophia-Antipolis).
\smallbreak
\textbf{2006-12 }PhD History and Archaeology, first level of distinction (''mention tr\'{e}s honorable avec les f\'{e}licitations du jury''). PhD title: ''Organisation spatiale et s\'{e}riation des gravures piquet\'{e}es du mont Bego'', Universit\'{e} Nice Sophia-Antipolis, UMR 7264 CEPAM-CNRS.
\smallbreak
\textbf{2005-6 }Master Research ('mention bien') "Sciences de l'Homme et de la Soci\'{e}t\'{e} Histoire et Arch\'{e}ologie", mention~: "Sciences des Mondes Pr\'{e}historiques, Antiques et M\'{e}di\'{e}vaux", Universit\'{e} Nice Sophia-Antipolis, CEPAM-CNRS UMR 7264.
\smallbreak
\textbf{2004-5 }DUT (University Diploma of Technology) Informatics Engineering (''mention assez bien''), Conservatoire National des Arts et M\'{e}tiers, Paris.
\smallbreak
\textbf{2003 } Topography diploma, Universitad de Ingenier\'{i}a, Lima, Peru.

\section*{PUBLICATIONS}

\begin{center}------------ Books ------------ \end{center}

\textbf{Huet T., 2017}, \href{http://www.prehistoire.org/shop_515-40342-0-0/m63-2017-les-gravures-piquetees-du-mont-bego-alpes-maritimes-organisation-spatiale-et-seriation-vie-iie-millenaire-av.-j.-c.-t.-huet.html}{\textit{Les gravures piquet\'{e}es du mont Bego (Alpes-Maritimes). Organisation spatiale et s\'{e}riation (6e -- 2e mill\'{e}naire av. J.-C.)}, M\'{e}moire de la Soci\'{e}t\'{e} Pr\'{e}historique Fran\c{c}aise (SPF) 63, 166 p.} 
\bigbreak
\begin{center}------------ Book chapters ------------\end{center}
\smallbreak
Alexander C., Maretta A., \textbf{Huet T.} and C. Chippindale, \textit{\textbf{in press}}, Rules of ordering and grouping in the 'pitoti', the later prehistoric rock-engravings of Valcamonica (BS), Italy: from solitary figures through clusters, graphic groups, and scenes to narrative, in I. Davidson and A. Nowell (eds), \textit{Making scenes: global perspectives on scenes in rock art}, London: Berghahn Books.
\smallbreak
\textbf{Huet T., 2018}, \href{https://hal.archives-ouvertes.fr/hal-01983284}{Une revue de l'iconographie du d\'{e}but du N\'{e}olithique \'{a} la fin de l'\^{a}ge du Bronze (ca. 5700-750 av. J.-C.) en France, \textit{in} Guilaine J. et D. Garcia (dir.), \textit{La Protohistoire de la France}, ed. Hermann, Paris, p. 221-249.}
\bigbreak
\begin{center}------------ Indexed journals ------------\end{center}
\smallbreak
Nieto-Espinet A., \textbf{Huet T.}, Trentacoste A., Guimaraes S., Orengo H. and Valenzuela-Lamas S., \textbf{\textit{2021}}, Resilience and livestock adaptations to demographic growth and technological change: A diachronic perspective from the Late Bronze Age to Late Antiquity in NE Iberia, \textit{PlosONE}, \href{https://doi.org/10.1371/journal.pone.0246201}{doi: 10.1371/journal.pone.0246201}
\smallbreak
Iba\~{n}ez J.J., Mu\~{n}iz J., \textbf{Huet T.}, Borrell Terra F.,  Santana Y., Teira Mayolini L.C. and R. Rosillo, \textbf{2020}, Flint Figurines in the Early Neolithic site of Kharaysin (Early 8th Millennium BC, Jordan), \textit{Antiquity Journal, 94, 376} p. 880-899, \href{https://doi.org/10.15184/aqy.2020.78}{doi: 10.15184/aqy.2020.78}
\smallbreak
Cicolani V. and \textbf{T. Huet, 2019}, Essai de mod\'{e}lisation des \'{e}changes et des r\'{e}seaux de circulation dans les Alpes centrales au premier \^{a}ge du Fer, \textit{in} "La conqu\^{e}te de la montagne : des premi\'{e}res occupations humaines \'{a} l'anthropisation du milieu", CTHS \'{e}ditions, \href{https://halshs.archives-ouvertes.fr/halshs-02314978/document}{https://halshs.archives-ouvertes.fr/halshs-02314978/document}
\smallbreak
\textbf{Huet T.,} \textbf{2018}, \href{http://archaeopress.com/ArchaeopressShop/Public/displayProductDetail.asp?id=\%7B2724F16C-FAC1-4987-8D1E-E85D9F94ACAD\%7D}{Geometric Graphs to Study Ceramic Decoration\textit{, in} M. Matsumoto and  E. Uleberg (eds), Exploring Oceans of Data, \textit{Proceedings of the 44${}^{nd\ }$Conference on Computer Applications and quantitative Methods in Archaeology (caa'16) }[2016], Oxford : Archaeopress Archaeology, 311-323.}
\smallbreak
\textbf{Huet T., 2016}, New perspectives on the chronology and the meaning of Mont Bego's rock-art (Alpes-Maritimes, France), \textit{Cambridge Archaeological Journal} 41, 1-23, doi: \href{https://doi.org/10.1017/s0959774316000524}{10.1017/s0959774316000524}
\smallbreak
\textbf{Huet T.} and N. Bianchi, \textbf{2016}, A study of the Roche de l'Autel's pecked engravings, Les Merveilles sector, Mont Bego area (Alpes-Maritimes, France), \textit{Journal of Archaeological Sciences: Reports} 5, 105-118, doi: \href{https://doi.org/10.1016/j.jasrep.2015.11.006}{10.1016/j.jasrep.2015.11.006}
\smallbreak
\textbf{Huet T.} and N. Bianchi, \textbf{2016}, \href{http://www.ccsp.it/web/infoccsp/bcsp/bcsp41_preview.pdf}{Reticolati, pelli e mappe topografiche, lo stato della ricerca al monte Bego, \textit{Bollettino del Centro Camuno di Studi Preistorici} 41, 31-43.}
\smallbreak
\textbf{Huet T., 2016}, \href{https://dialnet.unirioja.es/servlet/articulo?codigo=6028872}{S\'{e}riation des gravures piquet\'{e}es du mont Bego, \textit{Archeologia e Calcolatori} 27, 65-83.}
\smallbreak
\textbf{Huet T.,} \textbf{2015}, \href{https://www.insegnadelgiglio.it/wp-content/uploads/2015/01/APM_17_libro-anteprima.pdf}{Le incisioni a martellina del monte Bego: approcci geografici e quantitativi, \textit{Archeologia Postmedievale} 17, 329-338.}
\smallbreak
Saligny L., Granjon L., \textbf{Huet T.}, Simon G., Rodier X. and B. Lefebvre \textbf{2015}, \href{https://hal.archives-ouvertes.fr/halshs-01146871}{OH\_FET: A Computer Application for Analysing Urban Dynamics Over Long Time Spans, in\textit{ Proceedings of the 42${}^{nd\ }$Conference on Computer Applications and quantitative Methods in Archaeology, (caa'14) }[2014], eds\textit{. }Giligny F., Djindjian F., Costa L., Moscati P. et S. Robert, Oxford : Archaeopress Archaeology, 381-392.}
\smallbreak
\textbf{Huet T.} and C. Alexander, \textbf{2015}, M\'{e}thodes informatiques pour l'\'{e}tude des gravures rupestres~: les exemples du Valcamonica (Italie) et du mont Bego (France), in \textit{Recherches sur l'\^{a}ge du Bronze. Nouvelles approches et perspectives. Actes de la journ\'{e}e d'\'{e}tude de l'APRAB, Bulletin de l'APRAB, suppl. 1}, 15-29.
\smallbreak
Ouriachi M.-J., Favory F., Garmy P., Ouzoulias P., Pasqualini A., Christol M., \textbf{Huet T}., Nuninger L., Bertoncello F., H\"{a}ussler R.. \textbf{2014}, \href{https://www.persee.fr/doc/ran_0557-7705_2014_num_47_1_1897}{ArchaEpigraph : l'\'{e}pigraphie spatiale au service de l'\'{e}tude des dynamiques des territoires, in \textit{Revue arch\'{e}ologique de Narbonnaise}, tome 47, pp. 35-49.}
\smallbreak
\textbf{Huet T.,} \textbf{2014}, Use of quantitative methods to study an Alpine rock art site: the Mont Bego region", in \textit{Proceedings of the 40${}^{th}$ Conference on Computer Applications and quantitative Methods in Archaeology (caa'12) }[2012], eds. Earl G., Sly T., Chrysanthi A., Murrieta Flores P., Papadopoulos C., Romanowska I. et D. Wheatley, Southampton, UK, 26-30 March 2012, Amsterdam : Pallas Publications, , 584-591.
\smallbreak
\textbf{Huet T.,} \textbf{2014}, \href{https://www.persee.fr/doc/bspf_0249-7638_2013_num_110_1_14242}{Organisation spatiale et s\'{e}riation des gravures piquet\'{e}es du mont Bego~-- R\'{e}sum\'{e} de th\'{e}se, \textit{Bulletin de la Soci\'{e}t\'{e} Pr\'{e}historique Fran\c{c}aise} 110, 146-148.}

\subsection*{Scientific Conferences and seminars }
\begin{center}(\textbf{i}) international = 9 {\textbar} (\textbf{n}) national audience = 7 \end{center}
\smallbreak
\begin{center}------------ Organisation ------------\end{center}
\smallbreak
\textbf{2016 }S\'{e}minaire d'\'{e}quipe ASM-CNRS (UMR 5140), \'{e}quipe Soci\'{e}t\'{e}s de la Pr\'{e}histoire et de la Protohistoire, Analyser et interpr\'{e}ter les d\'{e}cors des c\'{e}ramiques pr\'{e}- et protohistoriques. Approches crois\'{e}es, co-organis\'{e} avec T. Lachenal et K. Peche-Quichilini, Universit\'{e} Paul-Val\'{e}ry, Montpellier, 27 May. (\textbf{n})
\smallbreak
\textbf{2014 }Session n$\mathrm{{}^\circ}$ 20 :\textit{ "(Re)building past networks: archaeological science, GIS and network analysis"}, co-organised with C. Alexander, S. Robert, E. Mermet, 42${}^{th}$ international congress of the \textit{Computer Applications and Quantitative Methods in Archaeology (caa) 2014}, Universit\'{e} Sorbonne, France, 22-25 April. (\textbf{i})
\bigbreak
\begin{center}------------ Communications ------------\end{center}
\smallbreak
\textbf{2019 }"Diacronia delle incisioni rupestri preistoriche e protostoriche della regione del monte Bego (Tenda, Alpi Marittime, Francia)", co-presented with J. Masson Mourey J. and N. Bianchi, LIV Riunione scientifica, Archeologia del cambiamento. Modelli, processi, adattamenti nella Preistoria e Protostoria, Roma, 23-26 October 2019. \textbf{(i)}
\smallbreak
\textbf{---  }"The North-western Impressed Wares: a chrono-cultural overview", co-presented with D. Binder, L. Angeli, L. Gomart, R. Maggi, C. Manen, I.M. Muntoni, E. Natali, C. Panelli, G. Radi, F. Radina, C. Tozzi, S. Tusa, 1st conference on the Early Neolithic of Europe (ENE2019), Barcelona, 6-9 November. \textbf{(i)}
\smallbreak
\textbf{---  }"L'Impresso-cardial du nord-ouest et ses rapports avec la "zone-source" : une synth\'{e}se chrono-culturelle", co-presented with D. Binder, L. Angeli, L. Gomart, R. Maggi, C. Manen, I. Maria Muntoni, E. Natali, C. Panelli, G. Radi, F. Radina, C. Tozzi, S. Tusa, S\'{e}ance de la Soci\'{e}t\'{e} Pr\'{e}historique Fran\c{c}aise, Nice, 18 - 20 March.\textbf{ (i)}
\smallbreak
\textbf{---  }"Territorialit\'{e}, mobilit\'{e}, interactions : apports crois\'{e}s des sous-syst\'{e}mes lithiques et c\'{e}ramiques", co-presented with D. Binder, C. De Stefanis, P. Fernandes, B. Gratuze, R. Maggi, C. Tozzi, S\'{e}ance de la Soci\'{e}t\'{e} Pr\'{e}historique Fran\c{c}aise, Nice, 18 - 20 March.\textbf{ (i)}
\smallbreak
\textbf{2018 }"Pastoral graffiti and ''protohistoric'' engravings in Mont Bego region: a study of marking practices over long time spans", co-presented with N. Magnardi, 20${}^{e}$ international rock-art congress organisation (ifrao), Darfo Boario Terme, 29 August -2 September.\textbf{ (i)}
\smallbreak
\textbf{2017 }"Las representaciones iconogr\'{a}ficas en Francia del Neol\'{i}tico antiguo a la fin del Edad del Bronce (5700-750 BC)", SEMINARIOS Grupo de investigaci\'{o}n "Arqueolog\'{i}a de las din\'{a}micas sociales", Instituci\'{o}n Mil\'{a} y Fontanals (IMF) - Consejo Superior de Investigaciones Cient\'{i}ficas (CSIC), Barcelone, 12 December. (\textbf{n})
\smallbreak
\textbf{---  } "Interactions culturelles au Premier \^{a}ge du Fer dans les Alpes occidentales : essai de mod\'{e}lisation des \'{e}changes par la th\'{e}orie des r\'{e}seaux~", co-presented with V. Cicolani, 142${}^{e}$ congr\'{e}s du \textit{Comit\'{e} des Travaux Historiques et Scientifiques} (cths), Pau, 24 -29 April.\textbf{ (n)}
\smallbreak
\textbf{2016 }"Geometrical and planar graphs in ancient iconography studies, a heuristic tool", Session \textit{Networking the past: Towards best practice in archaeological network science}, 44${}^{th}$ international congress of the \textit{Computer Applications and Quantitative Methods in Archaeology }(caa) \textit{2016}, University of Oslo, Sweden, 29 March-2 April. (\textbf{i})\textbf{}
\smallbreak
\textbf{---  }"L'\'{e}tude des d\'{e}cors c\'{e}ramiques figuratifs du Mailhac I et des d\'{e}cors apparent\'{e}s (Bronze final IIIb). Contextes, m\'{e}thodes et premiers r\'{e}sultats" journ\'{e}e th\'{e}matique Images et imaginaire \'{a} l'\^{A}ge du bronze en Europe, \textit{Association pour la Promotion de la Recherche sur l'\^{A}ge du Bronze} (APRAB), Mus\'{e}e des Antiquit\'{e}s Nationales, Saint-Germain-en-Laye, 4 March. (\textbf{n})
\smallbreak
\textbf{2015 }"Les gravures rupestres n\'{e}olithiques du mont Bego (Alpes-Maritimes) : identifications, comparaisons et significations suppos\'{e}es", \textit{L'art n\'{e}olithique du Proche-Orient \'{a} l'Europe : nouvelles, approches th\'{e}oriques et m\'{e}thodologiques}, round table of the \textit{Association pour le D\'{e}veloppement des Rencontres et des Echanges Universitaires et Culturels} (adreuc), Notre Dame de l'Abbaye, Carcassonne, 5 November. (\textbf{n})
\smallbreak
\textbf{---  }"Familles aristocratiques dans la cit\'{e} antique de N\^{i}mes : un exemple de mise en r\'{e}seaux des individus par les monuments \'{e}pigraphiques (Ier si\'{e}cle av. notre \'{e}re -- IIIe si\'{e}cle)" co-pr\'{e}sent\'{e}e avec M.-J. Ouriachi, 140${}^{th}$ congress of the \textit{Comit\'{e} des Travaux Historiques et Scientifiques} (cths), Reims, 27 April-2 May. (\textbf{n})
\smallbreak
\textbf{2014 }"Ancient pastoral paths in mount Bego rock art area" Session \textit{(Re)building past networks: archaeological science}, GIS and network analysis, 42${}^{th}$ international congress of the \textit{Computer Applications and Quantitative Methods in Archaeology }(caa) \textit{2014}, Universit\'{e} Sorbonne, France, 22-25 April. (\textbf{i})\textbf{}
\smallbreak
\textbf{---  }"OH-FET : a computer application to compare urban dynamics over long time spans (longue dur\'{e}e)", Session n$\mathrm{{}^\circ}$ 18, co-presented with L. Saligny, L. Granjon, B. Lefebvre, G. Simon, X. Rodier, 42${}^{th}$ international congress of the \textit{Computer Applications and Quantitative Methods in Archaeology }(caa)\textit{ 2014}, Universit\'{e} Sorbonne, France, 22-25 April. (\textbf{i})
\smallbreak
\textbf{---  }"M\'{e}thodes informatiques en art rupestre. Etudes de cas : le Valcamonica et le mont Bego", co-presented with C. Alexander, thematic day \textit{Recherches sur l'\^{A}ge du Bronze }of the\textit{ Association pour la Promotion de la Recherche sur l'\^{A}ge du Bronze }(aprab), Mus\'{e}e des Antiquit\'{e}s Nationales, Saint-Germain-en-Laye, 28 February. (\textbf{n})

\section*{FIELDWORK EXPERIENCE}

\textbf{2019 }Excavations of Nahal Efe (Negev, Israel), Israel Antiquities Authorities (IAA) / Instituci\'{o}n Mil\'{a} y Fontanals - Consejo Superior de Investigaciones Cient\'{i}ficas (IMF-CSIC), resp. J.K. Sharvit and F. Borrell, Ein Tamar, 29 April-25 may.
\smallbreak
\textbf{2014 }Inventory/Study of the archaeological artefacts of Ciari's shelter conserved in: Museo Civico -- Cuneo; Superintendencia -- Turin, in collaboration with N. Bianchi (Institute of Human Paleontology) and D. Binder (CEPAM-CNRS), 20-21 October.
\smallbreak
\textbf{---  }Paleoenvironmental coring, mont Bego archaeological site (Alpes-Maritimes, France), UMR 6249, Laboratoire Chrono-Environnement, dir. B. Vanni\'{e}re, 6---9 October.
\smallbreak
\textbf{---  }Photogrammetry and topography, Vert-la-Gravelle archaeological site (Marne, France), dir. R. Martineau, UMR 6298, ARTeHis, 15---30 August.
\smallbreak
\textbf{---  }Archaeological excavations, Ponteau-Gare archaeological site (Bouches-du-Rh\^{o}ne, France), dir. X. Margarit, SRA-PACA, 1---15 July.\smallbreak
\textbf{2013 }Aerial\textbf{ }photogrammetry and topography of the medieval Church of Saint Colomban Monastery archaeological site, Haute-Sa\^{o}ne, UMR 6298, ARTeHis -- NUI Galway, dir. S. Bully and E. Marron, 6-8 August.

\section*{INFORMATION TECHNOLOGIES RELEASES}

\subsection*{Softwares and Packages}

\begin{itemize}
  \item \textbf{2021 }\href{https://cran.r-project.org/web/packages/iconr/index.html}{\textbf{\textit{iconr}} package} \textsf{R} CRAN package giving concepts and computer tools to model Prehistorical iconographic composition with graph theory and GIS spatial database facilities (with Jose Pozo and Craig Alexander).
  \item \textbf{2014 }the \href{https://www.oxbowbooks.com/dbbc/caa2014-21st-century-archaeology.html/}{OH-FET application}, a \textsf{Python} app to compare urban dynamics over long time spans (with Ludovic Granjon and Laure Saligny, MSH de l'Université de Bourgogne).
\end{itemize}

\subsection*{Web applications}

\begin{itemize}
\item the \href{https://epispat.shinyapps.io/analyses_mult_5/}{EpiSpat application} RShiny interactive web application for multifactorial on-the-fly clustering of Roman epigraphic stelae
\item the \href{https://neolithic.shinyapps.io/Euroevol\_R/}{\includegraphics[scale=0.20]{"euroevol_R"}} and \href{https://neolithic.shinyapps.io/neonet2/}{\includegraphics[scale=0.04]{"neonet"}} RShiny interactive web applications for selecting, calibrating, summing and plotting radiocarbon dates. EUROEVOL\_R parses the \href{http://discovery.ucl.ac.uk/1469811/}{EUROEVOL database} (Institute of Archaeology, UCL). NeoNet database is on development (with Niccolo Mazucco).
\end{itemize}

\subsection*{Web docs}

\begin{itemize}
\item \href{https://neolithic.shinyapps.io/AbsoluteDating/}{Radiocarbon and dendrochronological dates}, a R + Shiny + Markdown document to illustrate computer-based applications and databases queries for absolute dating 
\item \href{https://zoometh.github.io/golasecca/}{Golasecca-net}, a R + Markdown document for archaeological data visualisation (maps, charts, spatial networks) of movable goods using plot\_ly and Leaflet packages, doi: \href{https://doi.org/10.5281/zenodo.4315328}{10.5281/zenodo.4315328}.
\end{itemize}
% \includegraphics{{mds}}

\section*{ACADEMIC AND RESEARCH PROGRAMS (2014-20)}

\subsection*{Teachings}

\textbf{2017 }"Introduction to Geographic Information Systems: QGIS", \textit{Grupo de Arqueologia de las Din\'{a}micas Sociale}s, Consejo Superior de Investigaciones Cientificas, Instituci\'{o}n Mil\'{a} i Fontanals (CSIC-IMF), 14-16 December (9 heures).\textbf{}
\smallbreak
\textbf{---  }"Les repr\'{e}sentations d'attelages \'{a} l'\^{a}ge du Bronze (Espagne, France, Italie), dans le cadre du \textit{S\'{e}minaire de formation doctorale du r\'{e}seau interdisciplinaire AniMed}, University seminar, Universit\'{e} Paul-Val\'{e}ry, 26 January (1 hour).
\smallbreak
\textbf{2016 }"M\'{e}thodes quantitatives en arch\'{e}ologie. L'approche processuelle : concepts, outils et cas d'\'{e}tude" Master 2 Arch\'{e}ologie, Sciences pour l'Arch\'{e}ologie, ASM-UMR 5140, University seminar, Universit\'{e} Paul-Val\'{e}ry, 5 December (4 hours).
\smallbreak
\textbf{2015 }"L'apparition des d\'{e}cors figuratifs du Mailhac I et des groupes apparent\'{e}s dans une Europe continentale largement aniconique (Bronze final IIIb, 950-750 av. J.-C.) : contextes, hypoth\'{e}ses et m\'{e}thodes" Master 1 Recherche, Protohistoire m\'{e}diterran\'{e}enne, ACTE -- UE 3-6, University seminar, Universit\'{e} de Bourgogne, 24 November (1 hour).
\smallbreak
\textbf{---  }"L'apport du SIG \'{a} l'\'{e}tude de l'art rupestre du Mont Bego (Alpes-Maritimes)" Master 2 Recherche et Professionnel, Arch\'{e}ologie des Soci\'{e}t\'{e}s et Territoires en France M\'{e}tropolitaine, M\'{e}thodes et approches nouvelles, LARA, CReAAH-UMR 6566, University seminar, Universit\'{e} de Nantes, Universit\'{e} de Rennes, 4 November (1 hour).
\smallbreak
\textbf{---  }"Les m\'{e}thodes quantitatives en arch\'{e}ologie : de la \textit{New Archaeology} au \textit{Project Mosul}" Master 2 Recherche et Professionnel, Arch\'{e}ologie des Soci\'{e}t\'{e}s et Territoires en France M\'{e}tropolitaine, S\'{e}minaire sp\'{e}cialis\'{e} : "M\'{e}thodes et approches nouvelles", LARA, CReAAH-UMR 6566, University seminar, Universit\'{e} de Nantes, Universit\'{e} de Rennes, 16 October (1 hour).
\smallbreak
\textbf{---  }"Data Paths", University workshop, Z\'{a}pado\v{c}esk\'{a} Univerzita (\textit{University of West Bohemia}), Pilsen, Czech Republic, 12 March, (1 hour).

\subsection*{Research Programs}

\subsubsection*{Academic committees}

\textbf{Member of the PhD steering committee} (2015-2020) of: Francis Bordas, University Toulouse 2, Doctoral section TESC, UMR TRACES, PhD title: "Les d\'{e}p\^{o}ts d'objets m\'{e}talliques au Bronze final 3 dans l'espace atlantique fran\c{c}ais (950-800 av J.-C.) : modalit\'{e}s de constitution des d\'{e}p\^{o}ts fran\c{c}ais de l'\'{e}tape de l'\'{e}p\'{e}e du type en langue de carpe."
\smallbreak
\textbf{Tutor of the Master 2  steering committee} (2018-2019) of: Marco Padovan, University Toulouse 2, "Formation aux outils de traitement 3D" dans le cadre du Master 2 de, Universit\'{a} degli studi di \textit{Ferrara} (Italia), Master 2 title: "The Fabric of the Baume de Monthiver"

\subsubsection*{Journal committees}

-  \textit{CAA Review College} member (2014 - ...)

-  Reviewer for the \textit{Bulletin de la Soci\'{e}t\'{e} Pr\'{e}historique Fran\c{c}aise} journal (2020-...)

-  Reviewer for the \textit{M@ppemonde} journal (2019-...)

\subsubsection*{Collective research projects}

-  Member of the \href{https://redneonet.com/}{NeoNet} collective research project, resp. J. Gibaja, M. Cubas (2020-...).

-  Former member of the \textit{Corpus des signes grav\'{e}s n\'{e}olithiques} collective research project, resp. S. Cassen.

-  Former member of the \textit{Information Spatiale et Arch\'{e}ologie} network (ISA). 

-  Former member of the "~Faire Science avec l'Incertitude en SHS~" work group, MSH Nice.

-  Former member of the \textit{Evolutions, transferts, inter-culturalit\'{e}s dans l'arc liguro-proven\c{c}al : Mati\'{e}res premi\'{e}res, productions et usages, du Pal\'{e}olithique sup\'{e}rieur \'{a} l'\^{a}ge du Bronze ancien} (ETICALP) collective research project (PCR), reps. D. Binder (2008 -- 14).

\section*{TRAINING COURSES ATTENDED}

\subsection*{Statistics }

\textbf{2014 }"\textsf{R} application", training of the \textit{ElementR} group, UMR 8504 G\'{e}ographie-cit\'{e}s, MSH de l'Universit\'{e} de Bourgogne (Dijon, France) 27-28 January.
\smallbreak
\textbf{2010 }"Describe and analyse multifactorial data", Permanent training of the CNRS Delegation C\^{o}te d'Azur (Sophia-Antipolis, France), 7-9 September.
\smallbreak
\textbf{---  }"Decide and predict with multifactorial data: discriminant analysis and regressions", Permanent training of the CNRS D\'{e}l\'{e}gation C\^{o}te d'Azur (Sophia-Antipolis, France), 16-18 June.
\smallbreak
\textbf{2009 }"Statistical processing of small samples", Permanent training of the CNRS Delegation C\^{o}te d'Azur (Sophia-Antipolis, France), 14-16 October.
\smallbreak
\textbf{---  }"Introduction to standard statistics", Permanent training of the CNRS Delegation C\^{o}te d'Azur (Sophia-Antipolis, France), 16-18 September.
\textbf{---  }"Statistical processing of design of experiments: analysis of variance", Permanent training of the CNRS Delegation C\^{o}te d'Azur (Sophia-Antipolis, France), 16-18 June.
\smallbreak

\subsection*{Network analysis}

\textbf{2018 }"MOSAIC${}^{NET\ }$2018: Networks in Archaeological Research", International Summer School at~Bibracte, Ecole Europ\'{e}enne de Protohistoire (Glux-en-Glenne, France), 8-13 October.

\subsection*{Agent-based modeling}

\textbf{2016 }"Agent-based modelling for archaeologists", workshop in parallel of the \textit{Computer Applications and Quantitative Methods in Archaeology }(caa) \textit{2016 }international colloquium (Oslo, Sweden), 28-29 March.

\subsection*{3D / Photogrammetry}

\textbf{2016 }"\textit{CloudCompare} - manipulation of 3D~clouds", workshop in parallel of the \textit{Journ\'{e}es informatique et arch\'{e}ologies de Paris} (jiap), Institut d'Art et d'Arch\'{e}ologie (Paris, France), 8 June.
\smallbreak
\textbf{---  }"\textit{MeshLab }- manipulation of 3D\textit{ }meshes", workshop in parallel of the \textit{Journ\'{e}es informatique et arch\'{e}ologies de Paris} (jiap), Institut d'Art et d'Arch\'{e}ologie (Paris, France), 7 June.
\smallbreak
\textbf{2013 }"Aerial photography by kite and helium balloon: shot, georeferencing and photogrammetry", CNRS Formations entreprises (Al\'{e}s, Berrias et Casteljau, France), 28-30 May.

\subsection*{Image / Shape analysis}

\textbf{2013 }"Detection of structures and spatio-morphological dynamics by image analyses: morpho-mathematics", training organised by ISA network, UMR 7264 CEPAM-CNRS and UMR 7300 ESPACE (Nice, France), 18-21 March.
\smallbreak
\textbf{---  }"ImageJ", Permanent training of the CNRS Delegation C\^{o}te d'Azur (Sophia-Antipolis, France), 11-13 February.

\subsection*{Archaeology}

\textbf{2017-18 } "Record and wear analysis of engraving technics", training at the IMF-CSIC (Barcelona, Spain), 1 September - 30 May.
\smallbreak
\textbf{2010 } "Traceology of pre- and protohistorical stone tools", UMR 7264 CEPAM-CNRS, Universit\'{e} Nice Sophia-Antipolis (Nice, France), 4-15 October.
\smallbreak
\textbf{2008 } "Statement of carvings in Valcamonica archaeological site", training of the Archaeoloical Cooperative ''Le Orme dell Uomo'' (Capo di Ponte, Italy), 1-15 July.
\smallbreak
\textbf{2007 } "Technics for the archaeological cast", Mus\'{e}e municipal d'arch\'{e}ologie de Cimiez (Nice, France), 3-15 December.

\section*{INFORMATIC TOOLS}

\subsection*{Software development} 

R development with continuous integration (GitHub, Travis-CI)

\subsection*{Programming~languages {\textbar} Databases development}

\textsf{R}, \textsf{Python}, \textsf{JavaScript}, SQL, markup languages (Markdown, XML, JSON, HTML, etc.)~\textbf{{\textbar}} PostgreSQL/PostGIS, FileMaker, SpatiaLite, Access

\subsection*{Web authoring framework for data science {\textbar} Web platforms}

RShiny, Rmarkdown, Sweave \textbf{{\textbar}} RPubs, Zenodo

\subsection*{Infography~{\textbar} Photogrammetry/3D~{\textbar} Desktop publishing}

ImageJ, ImageMagick, current CAD \textbf{{\textbar}} MeshLab, CloudCompare, PhotoScan \textbf{{\textbar}} \LaTeX, Windows Office.\textbf{}

\subsection*{Geographical Information Systems (GIS) {\textbar} Interactive maps {\textbar} Geopositionning and topography}

QGIS/GRASS, ArcGIS {\textbar} Leaflet {\textbar} Differential GPS, Total Station, Theodolite.

\subsection*{Agent-based modeling (ABM)}

NetLogo, RNetLogo

\section*{LANGUAGES}

French: native \\
English, Spanish: good \\
Italian, Catalan: reading, listening \\
German: beginner.

\section*{COMPLEMENTARY INFORMATIONS}

Driving licence (permis B), Diving level 1, Animation Capacity Diploma (BAFA)

\end{document}
